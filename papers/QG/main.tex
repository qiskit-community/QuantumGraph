\documentclass[conference]{IEEEtran}
\IEEEoverridecommandlockouts
% The preceding line is only needed to identify funding in the first footnote. If that is unneeded, please comment it out.
\usepackage{cite}
\usepackage{amsmath,amssymb,amsfonts}
\usepackage{algorithmic}
\usepackage{graphicx}
\usepackage{textcomp}
\usepackage{xcolor}
\def\BibTeX{{\rm B\kern-.05em{\sc i\kern-.025em b}\kern-.08em
    T\kern-.1667em\lower.7ex\hbox{E}\kern-.125emX}}
\begin{document}


\title{A quantum procedure for map generation}



\author{
\IEEEauthorblockN{James R. Wootton}
\IEEEauthorblockA{
\textit{IBM Quantum, IBM Research - Zurich}\\
Switzerland \\
jwo@zurich.ibm.com
}
}

\IEEEpubid{\begin{minipage}{\textwidth}\ \\[12pt]
978-1-7281-4533-4/20/\$31.00 \copyright 2020 IEEE
\end{minipage}}

\maketitle

\begin{abstract}


Software for near-term quantum computers must be tailored to the hardware on which it runs. For hardware in which single and two-qubit gates are the primitive operations, the coupling graph is the device is an important constraint. This describes the pairs of qubits on which two qubit gates can be directly applied. Given this constraint, the Pauli expectation values for pairs of qubits in the coupling map are the variables that can be most directly manipulated by the primitive operations. The analysis of these variables throughout the course of a circuit could therefore be highly useful in designing and benchmarking near-term quantum software. In this paper we consider several practical considerations pertinent to such a study, and discuss results from a 53 qubit device.





\end{abstract}

\begin{IEEEkeywords}
\end{IEEEkeywords}



\section{Introduction}



\section{Conclusions}



\section*{Acknowledgment}



\bibliographystyle{IEEEtran}
\bibliography{refs}


\end{document}
